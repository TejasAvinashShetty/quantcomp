\begingroup
%
\par Letting $I \otimes X^c$ denote controlled-NOT conditioned on the first qubit, the left circuit diagram is:
%
\begin{align*}
J &= \left( H \otimes H \right) \left( I \otimes X^c \right) \left( H \otimes H \right) \\
&= \frac12 \left( H \otimes H \right) \begin{pmatrix} 1 & 0 & 0 & 0 \\ 0 & 1 & 0 & 0 \\ 0 & 0 & 0 & 1 \\ 0 & 0 & 1 & 0 \end{pmatrix} \begin{pmatrix} 1 & 1 & 1 & 1 \\ 1 & -1 & 1 & -1 \\ 1 & 1 & -1 & -1 \\ 1 & -1 & -1 & 1 \end{pmatrix} \\
&= \frac14 \begin{pmatrix} 1 & 1 & 1 & 1 \\ 1 & -1 & 1 & -1 \\ 1 & 1 & -1 & -1 \\ 1 & -1 & -1 & 1 \end{pmatrix} \begin{pmatrix} 1 & 1 & 1 & 1 \\ 1 & -1 & 1 & -1 \\ 1 & -1 & -1 & 1 \\ 1 & 1 & -1 & -1 \end{pmatrix} \\
&= \begin{pmatrix} 1 & 0 & 0 & 0 \\ 0 & 0 & 0 & 1 \\ 0 & 0 & 1 & 0 \\ 0 & 1 & 0 & 0 \end{pmatrix} \\
&= X^c \otimes I
\end{align*}
%
which is the right diagram.
%
\par Let $B : \ket{0,1} \mapsto \ket{\pm}$ where $B = H$.
%
Under the $\ket{\pm}$ basis, the operation $X^c \otimes I$ becomes:
%
\begin{align*}
L &= \left( B \otimes B \right) J \left( B \otimes B \right)^{-1} \\
&= \left( B \otimes B \right) J \left( B \otimes B \right) \\
&= \left( H \otimes H \right) \left( X^c \otimes I \right) \left( H \otimes H \right) \\
&= I \otimes X^c
\end{align*}
%
In other words, CNOT conditioned on the second qubit in the $\ket{0, 1}$ basis is equivalent to CNOT conditioned on the first qubit in the $\ket{\pm}$ basis.
%
\endgroup
