\begingroup
\newcommand{\roto}[2]{R_{#1} \left( #2 \right)}
\newcommand{\expo}[1]{\exp \left( #1 \right)}
\newcommand{\coso}[1]{\cos \left( #1 \right)}
\newcommand{\sino}[1]{\sin \left( #1 \right)}
%
\par Some identities ($c_\theta = \cos \theta$, and $s_\theta = \sin \theta$):
%
\begin{enumerate}
\item $YZ = iX$, $XZ = -iY$, $XY = iZ$.
%
\item $\roto{i}{-a} = c_{\frac{a}{2}} I - i s_{\left( -\frac{a}{2} \right)} \sigma_i = c_{\frac{a}{2}} I + i s_{\frac{a}{2}} \sigma_i = \roto{i}{a}^* = \roto{i}{a}^{-1}$.
%
\item Letting $b = a/2$:
%
\begin{enumerate}
\item \begin{align*}
\roto{z}{a} Z \roto{z}{a}^* &= \left( c_b I - i s_b Z \right) Z \left( c_b I + i s_b Z \right) \\
&= c_b^2 Z + s_b^2 Z^3 + i c_b s_b Z^2 - i c_b s_b Z^2 \\
&= Z
\end{align*}
%
\item \begin{align*}
\roto{z}{a} X \roto{z}{a}^* &= c_b^2 X + s_b^2 ZXZ + i c_b s_b XZ - i c_b s_b ZX \\
&= \left( c_b^2 - s_b^2 \right) X + 2 i c_b s_b XZ \\
&= c_a X + s_a Y
\end{align*}
%
\item \begin{align*}
\roto{z}{a} Y \roto{z}{a}^* &= c_b^2 Y + s_b^2 ZYZ + i c_b s_b YZ - i c_b s_b ZY \\
&= \left( c_b^2 - s_b^2 \right) Y + 2 i c_b s_b YZ \\
&= c_a Y - s_a X
\end{align*}
%
\item \begin{align*}
\roto{y}{a} Z \roto{y}{a}^* &= \left( c_b I - i s_b Y \right) Z \left( c_b I + i s_b Y \right) \\
&= c_b^2 Z + s_b^2 YZY + i c_b s_b ZY - i c_b s_b YZ \\
&= \left( c_b^2 - s_b^2 \right) Z - 2 i c_b s_b YZ \\
&= c_a Z + s_a X
\end{align*}
\end{enumerate}
\end{enumerate}
%
\proplabel{Proposition}: Let $n$ be a unit vector of $\reals^3$, and $0 \le \theta \le 2 \pi$, $0 \le \phi \le \pi$ such that $n = \left( \sin \phi \cos \theta, \sin \phi \sin \theta, \cos \phi \right)$.
%
Then,
%
$$\roto{n}{\omega} = \roto{\left( \theta, \phi \right)}{\omega} = \roto{z}{\theta} \roto{y}{\phi} \roto{z}{\omega} \roto{y}{-\phi} \roto{z}{-\theta}$$.
%
\par \proof:
%
\begin{align*}
\Omega &= \roto{y}{\phi} \roto{z}{\omega} \roto{y}{\phi}^* \\
&= c_{\frac{\omega}{2}} I - i s_{\frac{\omega}{2}} \roto{y}{\phi} Z \roto{y}{\phi}^* \\
&= c_{\frac{\omega}{2}} I - i s_{\frac{\omega}{2}} \left[ c_\phi Z + s_\phi X \right]
\end{align*}
%
Expanding again:
%
\begin{align*}
\roto{z}{\theta} \Omega \roto{z}{\theta}^* &= c_{\frac{\omega}{2}} I - i s_{\frac{\omega}{2}} \roto{z}{\theta} \left[ c_\phi Z + s_\phi X \right] \roto{z}{\theta}^* \\
&= c_{\frac{\omega}{2}} I - i s_{\frac{\omega}{2}} \left[ c_\phi Z + s_\phi \roto{z}{\theta} X \roto{z}{\theta}^* \right] \\
&= c_{\frac{\omega}{2}} I - i s_{\frac{\omega}{2}} \left[ c_\phi Z + s_\phi \left( c_{\theta} X + s_{\theta} Y \right) \right] \\
&= c_{\frac{\omega}{2}} I - i s_{\frac{\omega}{2}} \left( n_z Z + n_x X + n_y Y \right) \\
&= \roto{n}{\omega}
\end{align*}
%
\hfill $\square$
%
\par By Exercise 2.72, a qubit state $\ket{\psi}$ represented by the Bloch vector $\lambda$ will have density matrix $\rho = \ket{\psi} \bra{\psi} = \left( I + \lambda \cdot \sigma \right) / 2$.
%
\begin{align*}
\roto{z}{a} \rho \roto{z}{a}^* &= \frac12 \left( I + \roto{z}{a} \left( \lambda \cdot \sigma \right) \roto{z}{a} \right) \\
&= \frac12 \left[ I + \lambda_x \left( c_a X + s_a Y\right) + \lambda_y \left( c_a Y - s_a X \right) + \lambda_z Z \right] \\
&= \frac12 \left[ I + \lambda_x \left( c_a - s_ a \right) X + \lambda_y \left( c_a + s_a \right) Y + \lambda_z \right] \\
&= \frac{I + \lambda' \cdot \sigma}{2}
\end{align*}
%
So the effect of $\roto{z}{a}$ is to z-rotate $\lambda$ by $a$.
%
Similarly, $\roto{y}{a}$ will y-rotate $\lambda$ by $a$.
%
Letting $R_{z, \theta} = \roto{z}{\theta}$, the action of $\roto{n}{\omega}$ on $\lambda$ will be:
%
\begin{align*}
\rho' &= \roto{\left( \theta, \phi \right)}{\omega} \rho \roto{\left( \theta, \phi \right)}{\omega}^* \\
&=
R_{z, \theta}
\left( R_{y, \phi}
\left( R_{z, \omega}
\left( R_{y, -\phi}
\left( R_{z, -\theta} \rho R_{z, \theta} \right)
R_{y, \phi} \right)
R_{z, -\omega} \right)
R_{y, -\phi} \right)
R_{z, -\theta}
\end{align*}
%
which is a rotation around $n$ by $\omega$.
\endgroup
