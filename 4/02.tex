\begingroup
\newcommand{\diag}[1]{\mathrm{diag} \left[ #1 \right]}
\newcommand{\diags}[2]{\mathrm{diag} \left[ #1, #2 \right]}
\newcommand{\expo}[1]{\exp \left( #1 \right)}
\newcommand{\coso}[1]{\cos \left( #1 \right)}
\newcommand{\sino}[1]{\sin \left( #1 \right)}

\par \begin{quote} Let $x$ be a real number and $A$ a matrix such that $A^2 = I$.
%
Show that $\expo{i A x} = I \cos x + i A \sin x$. \end{quote}
%
\par In order for the operator function to make sense, $A$ must be normal operator (cf. Section 2.1.8).
%
However, being a self-inverse doesn't imply being normal.
%
For instance, $\begin{pmatrix} -1 & 1 \\ 0 & 1 \end{pmatrix}^2 = 1$ but isn't normal.
%
\par Let's assume that $A$ is normal as well.
%
By the Cayley-Hamilton theorem, $A$ has eigenvalues $\pm 1$ and so $A = U \diags{1}{-1} U^*$.
%
Then,
%
\begin{align}
\expo{i A x} &= U \diags{\expo{i x}}{\expo{-i x}} U^* \\
&= U \diags{\coso{x}}{\cos{-x}} U^* + U \diags{i \sino{x}}{i \sino{-x}} U^* \\
&= U \diag{\coso{x}} U^* + i U \diags{\sino{x}}{-\sino{x}} U^* \\
&= \coso{x} U I U^* + i \sino{x} U \diags{1}{-1} U^* \\
&= \coso{x} I + i \sino{x} A
\end{align}
\endgroup
