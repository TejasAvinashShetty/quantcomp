\documentclass[all.tex]{subfiles}
\begin{document}
\ignorespaces
\parindent 0pt
%\par For $A \in \complex^{N \times N}$, let $B = A + A^*$ and $C = A - A^*$.
%Then $A = B + C$. Also, $B^* = \left( A + A^* \right)^* = A^* + A = B$ and $C^*
%= \left( A - A^* \right)^* = A^* - A = -C^*$.
%
%\par Now let $A$ be a positive operator: $0 \le \braket{x|A|x} \in \reals$.
%Then:
%
%\begin{align}
%\braket{x|A^*|x} &= \braket{x|B^*|x} + \braket{x|C^*|x} \\
%&= \braket{x|B|x} + 
%\end{align}
\begin{quote}
``A special subclass of Hermitian operators is extremely important. This is the
positive operators. A positive operator A is defined to be an operator such that
for any vector $\ket{v}$, $\left( \ket{v}, A \ket{v} \right)$ is a real,
non-negative number."
\end{quote}
\par If $A \in \complex^{n \times n}$ is a positive operator, then $k =
\braket{x|A|x} \ge 0$ is real and so, trivially, $k = k^* = \braket{x|A^*|x} =
\braket{x|A|x}$, thus $A = A^*$.

\par Note that this argument fails for $B \in \reals^{n \times n}$, since
$\braket{x|B|x} = x^T B x = \braket{x|C|x}$ does not imply that $B = C$; $B$
could, for instance, be anti-symmetric.
\end{document}

