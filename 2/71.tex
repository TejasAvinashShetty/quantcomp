\begingroup
\newcommand{\tr}[1]{\mathrm{tr} \left[ #1 \right]}
%
\par In an orthnormal basis $\{\ket{i}\}$,
%
\begin{align}
\tr{\ket{\alpha} \bra{\beta}}
&= \tr{\left( \sum_i{a_i \ket{i}} \right) \left( \sum_j{b_j^* \bra{j}} \right)} \\
&= \tr{\sum_{i, j}{a_i b_j^* \ket{i} \bra{j}}} \\
&= \sum_{i, j}{a_i b_j^* \tr{\ket{i} \bra{j}}} \\
&= \sum_{i, j}{a_i b_j^* \delta_{ij}} = \braket{\beta|\alpha}
\end{align}
%
\par Now let $\rho = \sum_i{p_i \ket{\psi_i} \bra{\psi_i}}$ be a density operator where $\{\ket{\psi_i}\}$ are all normal but may not necessarily be orthogonal to each other.
%
Then, $\rho^2 = \sum_{i, j}{p_i p_j \braket{\psi_i|\psi_j} \ket{\psi_i} \bra{\psi_j}}$ and:
%
\begin{align}
\tr{\rho^2}
&= \sum_{i, j}{p_i p_j \braket{\psi_i|\psi_j} \tr{\ket{\psi_i} \bra{\psi_j}}} \\
&= \sum_{i \ne j}{p_i p_j \braket{\psi_i|\psi_j} \tr{\ket{\psi_i} \bra{\psi_j}}} + \sum_k{p_k^2 \tr{\ket{\psi_k} \bra{\psi_k}}} \\
&= \sum_{i \ne j}{p_i p_j \abs{\braket{\psi_j|\psi_i}}}^2 + \sum_k{p_k^2 \braket{\psi_k|\psi_k}} \\
&\le \sum_{i \ne j}{p_i p_j} + \sum_k{p_k^2} \\
&= \left( \sum_i{p_i} \right)^2 = 1
\end{align}
%
\par In a pure state, $p_r = 1$ for some fixed $r$ and necessarily, all other $p_i = 0$ (thus, $p_i p_r = \delta_{ir}$). Then:
%
\begin{align}
\tr{\rho^2} &= \sum_{i \ne j}{p_i p_j \abs{\braket{\psi_j|\psi_i}}}^2 + \sum_k{p_k^2 \braket{\psi_k|\psi_k}} \\
&= \sum_k{p_k^2 \braket{\psi_k|\psi_k}} \\
&= p_r^2 \braket{\psi_r|\psi_r} = 1^2 \cdot 1 = 1
\end{align}
%
\par Conversely, suppose that $\tr{\rho^2} = 1$. Then,
%
\begin{align}
\sum_{i \ne j}{p_i p_j \abs{\braket{\psi_j|\psi_i}}}^2 + \sum_k{p_k^2
\braket{\psi_k|\psi_k}} &= 1 \\
\sum_{i \ne j}{p_i p_j \abs{\braket{\psi_j|\psi_i}}}^2 + \sum_k{p_k^2} &= \left(
\sum_k{p^k} \right)^2 \\
&= \sum_{i \ne j}{p_i p_j} + \sum_k{p_k^2} \\
\sum_{i \ne j}{p_i p_j \abs{\braket{\psi_j|\psi_i}}}^2 &= \sum_{i \ne j}{p_i
p_j} \\
p_i p_j \abs{\braket{\psi_j|\psi_i}}^2 &= p_i p_j, \quad i \ne j
\end{align}
%
Suppose that there exists a pair $r \ne s$ such that $p_r p_s > 0$. Then:
%
$$
p_r p_s \abs{\braket{\psi_s|\psi_r}}^2 = p_r p_s \Rightarrow
\abs{\braket{\psi_s|\psi_r}}^2 = 1 \Rightarrow \ket{\psi_r} = \ket{\psi_s}
\Rightarrow r = s
$$
%
leading to a contradiction.
%
So the product of probabilities for every pair of distinct states must be zero, which means that there can be at most one non-zero $p_r$.
%
They cannot all be zero either, because $\sum_k{p_k} = 1$.
%
Thus, $\sum_k{p_k} = p_r = 1$.
\endgroup
