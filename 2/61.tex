\par Using $V$ as defined in Exercise 60, $\mathrm{p} \left(+1 ; \ket{0} \right)
= \braket{0|P_+|0} = \frac{1}{2} \braket{0 | I + V | 0} = \frac{1}{2} \left( 1 +
v_2 \right)$. If $+1$ is gotten, then the post-measurement state will be

$$
\frac{P_+ \ket{0}}{\sqrt{\braket{0|P_+|0}}} = \frac{\left( 1 + v_2 \right)
\ket{0} + c \ket{1}}{2 \sqrt{\frac{1 + v_2}{2}}} = \frac{\left( 1 + v_2 \right)
\ket{0} + c \ket{1}}{\sqrt{2 \left( 1 + v_2 \right)}}
$$
