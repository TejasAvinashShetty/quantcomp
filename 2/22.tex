\documentclass[all.tex]{subfiles}
\begin{document}
\ignorespaces
\parindent 0pt

\par Let $A = A^*$, $A \ket{x} = \lambda_x \ket{x}$, and $A \ket{y} = \lambda_y
\ket{y}$ be distinct eigenvectors (and hence $\lambda_x \ne \lambda_y$). Then
$\braket{x|A|y} = \braket{x|\lambda_y|y} = \lambda_y \braket{x|y}$. But also,
$\braket{x|A|y} = \braket{x|A^*|y} = \lambda_x \braket{x|y}$. Subtracting,

\begin{align}
\lambda_x \braket{x|y} - \lambda_y \braket{x|y} &= 0 \\
\left( \lambda_x - \lambda_y \right) \braket{x|y} &= 0 \\
\braket{x|y} &= 0
\end{align}
\end{document}
