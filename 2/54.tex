\documentclass[all.tex]{subfiles}
\begin{document}
\ignorespaces
\parindent 0pt

\par If $A$ and $B$ are Hermitian and commute then by the simultaneous
diagonalization theorem, there exists an orthonormal basis $\ket{i}$ such that
$A = \sum_i{\lambda_i \ket{i} \bra{i}}$ and $B = \sum_i{\gamma_i \ket{i}
\bra{i}}$, with $\lambda_i, \gamma_i \in \reals$. Then:

\begin{align}
e^A e^B &= \left( \sum_i{e^{\lambda_i} \ket{i} \bra{i}} \right) \left(
\sum_i{e^{\gamma_i} \ket{i} \bra{i}} \right) \\
&= \sum_i{\sum_j{e^{\lambda_i + \gamma_j} \ket{i} \bra{i|j} \bra{j}}} \\
&= \sum_i{\sum_j{e^{\lambda_i + \gamma_j} \delta_{ij} \ket{i} \bra{j}}} \\
&= \sum_i{e^{\lambda_i + \gamma_j} \ket{i} \bra{i}} \\
&= \exp \left( {\sum_i{\left( \lambda_i + \gamma_j \right) \ket{i} \bra{i}}}
\right) \\
&= \exp \left( {\sum_i{\lambda_i \ket{i} \bra{i}} + \sum_i{\gamma_i \ket{i}
\bra{i}}} \right) \\
&= e^{A + B}
\end{align}
\end{document}
